\documentclass[11pt]{article}

\usepackage{amsmath,amssymb,amsfonts}
\usepackage{dsfont}
\usepackage{listings}
\usepackage{xcolor}
\usepackage{graphicx}
\usepackage{bbm}
\usepackage{float}
%\usepackage{hyperref}
\usepackage{subfig}
\usepackage[breaklinks=true,colorlinks,citecolor=black,linkcolor=gfGG,urlcolor=gfFP]{hyperref}
% http://www.colourlovers.com/palette/92095/Giant_Goldfish
\definecolor{gfCD}{HTML}{69D2E7} % Cloudless Day
\definecolor{gfSP}{HTML}{A7DBD8} % Sunken Pool
\definecolor{gfBS}{HTML}{E0E4CC} % Beach Storm
\definecolor{gfGG}{HTML}{F38630} % Giant Goldfish
\definecolor{gfFP}{HTML}{FA6900} % Food Pills

\definecolor{codegreen}{rgb}{0,0.6,0}
\definecolor{codegray}{rgb}{0.5,0.5,0.5}
\definecolor{codepurple}{rgb}{0.58,0,0.82}
\definecolor{backcolour}{rgb}{0.95,0.95,0.92}
\lstdefinestyle{mystyle}{
    backgroundcolor=\color{backcolour},
    commentstyle=\color{codegreen},
    keywordstyle=\color{magenta},
    numberstyle=\tiny\color{codegray},
    stringstyle=\color{codepurple},
    basicstyle=\ttfamily\footnotesize,
    breakatwhitespace=false,
    breaklines=true,
    captionpos=b,
    keepspaces=true,
    numbers=left,
    numbersep=5pt,
    showspaces=false,
    showstringspaces=false,
    showtabs=false,
    tabsize=2
}

\setlength{\topmargin}{-.5in} \setlength{\textheight}{9.25in}
\setlength{\oddsidemargin}{0in} \setlength{\textwidth}{6.8in}

%\newcommand*{\SOLVE}{}%

\renewcommand{\vec}[1]{\mbox{\boldmath$#1$}}
\newcommand{\mm}[1]{\mathbf{#1}}

\newcounter{ProblemNum}
\newcounter{SubProblemNum}[ProblemNum]

\renewcommand{\theProblemNum}{\arabic{ProblemNum}}
\renewcommand{\theSubProblemNum}{\alph{SubProblemNum}}

\newcommand*{\anyproblem}[1]{\section*{#1}}
\newcommand*{\problem}[1]{\stepcounter{ProblemNum} %
   \anyproblem{Problem \theProblemNum \; (#1 points)}}
\newcommand*{\soln}[1]{\subsection*{#1}}
\newcommand*{\solution}{\soln{Solution}}
\newenvironment{solutions}
  {\section[Solution]{\textcolor{red}{Solution}}\color{red}}
  {\normalcolor}
\renewcommand*{\part}{\stepcounter{SubProblemNum} %
  \soln{Part (\theSubProblemNum)}}
\renewcommand{\theenumi}{(\alph{enumi})}
\renewcommand{\labelenumi}{\theenumi}
\renewcommand{\theenumii}{\roman{enumii}}
\let\endsection\relax
\let\endsubsection\relax

\graphicspath{
{.}
}
\lstset{style=mystyle}

\begin{document}

\Large
\noindent{\bf CS4851/6851 IDL\:Homework 5 \hfill \today}
\medskip\hrule

\vspace{20pt}

Note: All coding problems to be submited with Github Link. Do not Upload the files/folder. Use git commands only.

Note: this is the distribution of questions:
\begin{enumerate}

  \item Question 1 to Question 2: Required for everyone.
  \item Question 3 to Question 4:  Bonus question for both Graduate Students and Undergraduate Students
\end{enumerate}


\problem{10}
For each of the following norms, explain what properties will they favor when used in reconstruction error: $L0$, $L1$, and $L2$
\problem{30}
Given a set of contrast images with sharp geometric edges (e.g. lithography masks for chip production) write down a formulation for reconstruction error that would work best. Justify your choice.

\problem{20}
Given a set of images of wild life taken in their natural habitat write down a formulation for reconstruction error that would work best. Justify your choice.


\noindent\rule[0.5ex]{0.45\linewidth}{1pt} Bonus for both  undergraduates and graduates beyond this line.


\problem{40} Given distributions p and q. If q is parameterized by $\theta$, how would you choose the value for $\theta$ to make q closest to p among all possible q's.
\begin{enumerate}
  \item Write down formulation of how would you measure the closeness of $q$ to $p$.
  \item Explain what you would do to maximize this closeness (i.e. make $q$ and $p$ maximally close, or minimally different or divergent)
\end{enumerate}


\problem{40} Write a report on one of the following topics related to GANS: 

\begin{enumerate}
  \item  InfoGAN {https://arxiv.org/abs/1606.03657}
  \item CycleGAN {https://arxiv.org/pdf/1703.10593.pdf}
\end{enumerate}

% \vspace{2px}
\end{document}

%%% Local Variables:
%%% mode: latex
%%% TeX-master: t
%%% End:  